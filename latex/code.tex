\documentclass[polish, 11pt]{article}

\usepackage[a4paper, margin=25mm]{geometry}
\usepackage{babel}
\usepackage{polski}
\usepackage[utf8]{inputenc}
\usepackage[T1]{fontenc}
\usepackage{lmodern}
\usepackage{graphicx}
\usepackage{pdfpages}
\usepackage{listings}
\usepackage{float}
\usepackage{xcolor}
\usepackage{gasasm}
\usepackage{hyperref}
\hypersetup{
    colorlinks=true,
    linkcolor=blue,
    filecolor=magenta,      
    urlcolor=cyan,
}
\graphicspath{ {images/} }
\lstset{language=[GAS]Assembler}


\begin{document}

\vspace{9cm}
\section{Kod źródłowy}
Lisint kodu z etapu 1
\begin{lstlisting}
#include <Wire.h>
#include <SFE_BMP180.h>
#include "DHT.h"
#include <BH1750.h>

#define ALTITUDE 85.0 
#define DHTPIN 7 
#define DHTTYPE DHT11   // DHT 11

SFE_BMP180 pressure;
DHT dht(DHTPIN, DHTTYPE);
BH1750 lightMeter;

void setup() {
  Serial.begin(9600);
  Serial.println(F("Test polaczenia z DHT rozpoczety"));
  Serial.println(F("Test polaczenia z BH1750 rozpoczety"));
  Serial.println(F("Test polaczenia z BMP180 rozpoczety"));
  
  Wire.begin();
  lightMeter.begin();
  dht.begin();
  pressure.begin();
}

void loop() {
  char status;
  double T,P,p0;
 
  status = pressure.startTemperature();
  if (status != 0)
  {
    delay(status);
    status = pressure.getTemperature(T);
    if (status != 0)
    {
     //Wypisz pomiary
      Serial.print("Temperatura: ");
      Serial.print(T,2);
      Serial.print(" deg C  ||    ");

        float h = dht.readHumidity();
        // sprawdzenie czy wczytanie danych przebieglo pomyslnie
        if (isnan(h)) {
          Serial.println(F("Nie udalo sie odczytac danych z DHT11!"));
          return;
        }
        Serial.print(F("Wilgotnosc powietrza: "));
        Serial.println(h);
      
      status = pressure.startPressure(3);
      if (status != 0)
      {
        delay(status);
        status = pressure.getPressure(P,T);
        if (status != 0)
        {
          // Wypisz pomiary:
          Serial.print("Cisnienie: ");
          Serial.print(P,2);
          Serial.print(" hPa ");

          p0 = pressure.sealevel(P,ALTITUDE); 
          Serial.print("   ||   Cisnienie ndmp: ");
          Serial.print(p0,2);
          Serial.println(" hPa");
                    
          //Wczytywanie natezenia
          float lux = lightMeter.readLightLevel();
            if (isnan(lux)) {
            Serial.println(F("Nie odczytano danych z BH1750!"));
            return;
          }
          Serial.print("Natezenie swiatla: ");
          Serial.print(lux);
          Serial.println(" lx");
          Serial.println();
          Serial.println();
        }
        else Serial.println("Nie udalo sie odczytac danych z BMP180!\n");
      }
      else Serial.println("Nie udalo sie odczytac danych z BMP180!\n");
    }
    else Serial.println("Nie udalo sie odczytac danych z BMP180\n");
  }
  else Serial.println("Nie udalo sie odczytac danych z BMP180!\n");

  //opoznienie pomiedzy pomiarami
  delay(5000);
}
\end{lstlisting}

\vspace{2cm}
Listing Kodu z dodanym modułem wifi:
\begin{lstlisting}
#include "WiFiEsp.h"
#include <Adafruit_BMP085.h>
#include "DHT.h"
//#include <BH1750.h>

// Piny 2 i 3 przy poborze danych
#ifndef HAVE_HWSERIAL1
#include "SoftwareSerial.h"
SoftwareSerial Serial1(2, 3); // RX, TX
#endif

Adafruit_BMP085 bmp;
#define ALTITUDE 85.0 

//Pin 7 przy poborze danych z DHT11
#define DHTPIN 7  
#define DHTTYPE DHT11
DHT dht(DHTPIN, DHTTYPE);

//BH1750 lightMeter;

char ssid[] = "StacjaPogodowa"; // nazwa sieci SSID
char pass[] = "12345678"; // haslo
int status = WL_IDLE_STATUS; // status WiFi
int reqCount = 0; // liczba otrzymanych requestow
int ledStatus = LOW;
WiFiEspServer server(80);

// RingBuffer pomaga zwiekszyc szybkość w przesyle danych
RingBuffer buf(8);

void setup()
{
  pinMode(LED_BUILTIN, OUTPUT);  
  Serial.begin(115200); // inicjalizuje polaczenie szeregowe z Arduino
  Serial1.begin(9600); // inicjalizuje polaczenie szeregowe z modulem WiFi
  
  Wire.begin(); 
  bmp.begin();
  dht.begin();
  lightMeter.begin();
  
  WiFi.init(&Serial1); // inicjalizacja modulu ESP

  Serial.println(ssid);

  // access point
  status = WiFi.beginAP(ssid, 10, pass, ENC_TYPE_WPA2_PSK);

  Serial.println("Access point ruszyl");
  printWifiStatus();
  
  // serwer uruchomiony na porcie 80
  server.begin();
  Serial.println("Serwer ruszył");
}

float lux;
float h;
char statuss;
double T,P,p0;

void loop()
{
  
  WiFiEspClient client = server.available();
  startMeasure();
  if (client) {             
    Serial.println("Nowy klient");
    buf.init();               
    while (client.connected()) {
      if (client.available()) 
        char c = client.read();
        buf.push(c);

        // zakończenie HTTP requestu to dwa znaki nowej linii
        if (buf.endsWith("\r\n\r\n")) {
          sendHttpResponse(client);
          break;
        }
        // Sprawdzenie czy zadaniem od klienta nie bylo wlaczenie lampki
        if (buf.endsWith("GET /H")) {
          Serial.println("LED wlaczony");
          ledStatus = HIGH;
          digitalWrite(LED_BUILTIN, HIGH); 
        }
        else if (buf.endsWith("GET /L")) {
          Serial.println("LED wylaczony");
          ledStatus = LOW;
          digitalWrite(LED_BUILTIN, LOW); 
        }
      }
    }
    
    // czas dla przegladarki na odbior danych
    delay(10);

    // close the connection
    client.stop();
    Serial.println("Klient rozlaczony");
  }
}

void sendHttpResponse(WiFiEspClient client)
{
  client.println("HTTP/1.1 200 OK");
  client.println("Content-type:text/html");
  client.println();
  
  // Zawartość stronki z danymi
  client.println("<!DOCTYPE HTML>");
  client.println("<html>");
  client.println("<body style=\"background-color:powderblue;\">");
  client.println("<h1 style=\"font-size:5vw;\">Witaj w moim domu:)</h1>");
  client.println("<h2 style=\"font-size:3vw;\">Temperatura: ");
  client.print(T);
  client.print(" &degC<br>Wilgotnosc: ");
  client.print(h);
  client.print(" %<br>Cisnienie: ");
  client.print(P);
  client.print(" hPa<br>Natezenie swiatla: ");
  client.print(lux);
  client.println(" lx<br><br>");  
  
  client.print("LED: ");
  client.print(ledStatus);
  client.println("<br>Wejdz <a href=\"/H\">tu</a> by wlaczyc (1)<br>");
  client.println("Wejdz <a href=\"/L\">tu</a> by wylaczyc (0)</h2><br>");
  client.println("</body>");
  client.print("</html>\r\n");
  
  // Odpowiedz protokolu HTTP to znak nowej linii
  client.println();
}

void printWifiStatus()
{
  // Wypisanie adresu IP
  IPAddress ip = WiFi.localIP();
  Serial.print("IP Address: ");
  Serial.println(ip);

  // Informacje na temat gdzie przejsc, by zobaczyc strone
  Serial.println();
  Serial.print("By zobaczyć strone, polacz sie z siecia");
  Serial.print(ssid);
  Serial.print(" i otworz przegladarke na http://");
  Serial.println(ip);
  Serial.println();
}

void startMeasure()
{
  lux = lightMeter.readLightLevel();
    if (isnan(lux)) {
    Serial.println(F("Nie udalo sie odczytac danych z czujnika BH1750"));
    lux = -1;
    }
  
  T =  bmp.readTemperature();
  if (isnan(T)) {
    Serial.println(F(""Nie odczytano danych z BMP180 - temperatura!"));
    T = -1;
  }
  
  P = bmp.readPressure();
  if (isnan(P)) {
    Serial.println(F(""Nie odczytano danych z BMP180 - cisnienie!"));
    P = -1;
  } P=P/100;
  
  h = dht.readHumidity();
  if (isnan(h)) {
      Serial.println(F(""Nie odczytano danych z DHT11 - wilgotnosc!"));
      h = -1;
  }
}

\end{lstlisting}

\end{document}